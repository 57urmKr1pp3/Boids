\documentclass[a4paper, 12pt]{article}
\usepackage{blindtext}
\begin{document}
	\begin{titlepage}
		\title{\Large{\textbf{\underline{Simulation von Boids nach der Idee von Craig Reynolds}}}}
		\author{Oliver Fritzler}
		\date{\today}
		\maketitle
	\end{titlepage}
	\title{\Large{\textbf{\underline{Inhaltsverzeichnis}}}}
	\tableofcontents
	\newpage
	\section{Abbildungsverzeichnis}
	\newpage
	\section{Quellennachweis}
	\newpage
	\section{UML-Diagramm}
	\newpage
	\section{Grundidee}
	\subsection{Allgemein}
		Boids sind Objekte die sich in einem Raum bewegen. Dabei verfolgen sie drei Grundregeln:
		\begin{itemize}
			\item\underline{"Seperation"}\linebreak
			Diese Regel besagt, dass jeder einzelne Boid versucht, keinen anderen Boid zu treffen. 
			\item\underline{"Alignement"}\linebreak
			Diese Regel besagt, dass jeder Boid versucht, in die selbe Richtung wie ein anderer sich zu bewegen. Dadurch entsteht ein sogenannter "Flock" also ein Schwarm von Boids.
			\item\underline{"Coheseion"}\linebreak
			Diese Regel besagt, dass die Boids in die Mitte des Schwarms steuern.
		\end{itemize}
	\subsection{Umgebung}
		Die Programmiersprache \emph{\textbf{Python}} war vorgegeben, jedoch war die Bibliothek zur grafischen Darstellung frei überlassen. Aufgrund der eindeutigen und verständlichen Syntax, fiel die Auswahl auf die \emph{\textbf{ursina engine}}. Diese basiert auf der \emph{\textbf{panda3D engine}}. Ursina ermöglicht es neben vorgegebenen Modellen auch blender-Dateien als Modelle für die einzelnen Objekte zu verwenden. 
	\newpage
	\section{wichtigste Funktionen}
	\newpage
\end{document}
